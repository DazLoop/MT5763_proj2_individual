\documentclass[]{article}
\usepackage{lmodern}
\usepackage{amssymb,amsmath}
\usepackage{ifxetex,ifluatex}
\usepackage{fixltx2e} % provides \textsubscript
\ifnum 0\ifxetex 1\fi\ifluatex 1\fi=0 % if pdftex
  \usepackage[T1]{fontenc}
  \usepackage[utf8]{inputenc}
\else % if luatex or xelatex
  \ifxetex
    \usepackage{mathspec}
  \else
    \usepackage{fontspec}
  \fi
  \defaultfontfeatures{Ligatures=TeX,Scale=MatchLowercase}
\fi
% use upquote if available, for straight quotes in verbatim environments
\IfFileExists{upquote.sty}{\usepackage{upquote}}{}
% use microtype if available
\IfFileExists{microtype.sty}{%
\usepackage{microtype}
\UseMicrotypeSet[protrusion]{basicmath} % disable protrusion for tt fonts
}{}
\usepackage[margin=1in]{geometry}
\usepackage{hyperref}
\hypersetup{unicode=true,
            pdftitle={180029941},
            pdfauthor={180029941},
            pdfborder={0 0 0},
            breaklinks=true}
\urlstyle{same}  % don't use monospace font for urls
\usepackage{graphicx,grffile}
\makeatletter
\def\maxwidth{\ifdim\Gin@nat@width>\linewidth\linewidth\else\Gin@nat@width\fi}
\def\maxheight{\ifdim\Gin@nat@height>\textheight\textheight\else\Gin@nat@height\fi}
\makeatother
% Scale images if necessary, so that they will not overflow the page
% margins by default, and it is still possible to overwrite the defaults
% using explicit options in \includegraphics[width, height, ...]{}
\setkeys{Gin}{width=\maxwidth,height=\maxheight,keepaspectratio}
\IfFileExists{parskip.sty}{%
\usepackage{parskip}
}{% else
\setlength{\parindent}{0pt}
\setlength{\parskip}{6pt plus 2pt minus 1pt}
}
\setlength{\emergencystretch}{3em}  % prevent overfull lines
\providecommand{\tightlist}{%
  \setlength{\itemsep}{0pt}\setlength{\parskip}{0pt}}
\setcounter{secnumdepth}{0}
% Redefines (sub)paragraphs to behave more like sections
\ifx\paragraph\undefined\else
\let\oldparagraph\paragraph
\renewcommand{\paragraph}[1]{\oldparagraph{#1}\mbox{}}
\fi
\ifx\subparagraph\undefined\else
\let\oldsubparagraph\subparagraph
\renewcommand{\subparagraph}[1]{\oldsubparagraph{#1}\mbox{}}
\fi

%%% Use protect on footnotes to avoid problems with footnotes in titles
\let\rmarkdownfootnote\footnote%
\def\footnote{\protect\rmarkdownfootnote}

%%% Change title format to be more compact
\usepackage{titling}

% Create subtitle command for use in maketitle
\newcommand{\subtitle}[1]{
  \posttitle{
    \begin{center}\large#1\end{center}
    }
}

\setlength{\droptitle}{-2em}

  \title{180029941}
    \pretitle{\vspace{\droptitle}\centering\huge}
  \posttitle{\par}
    \author{180029941}
    \preauthor{\centering\large\emph}
  \postauthor{\par}
      \predate{\centering\large\emph}
  \postdate{\par}
    \date{06/11/2018}


\begin{document}
\maketitle

\subsection{\texorpdfstring{\pagebreak  }{}}\label{section}

\subsection{Introduction}\label{introduction}

The present report aims to build a model that predicts Oxygen intake
rates (a measure of aerobic fitness) supported on a series of
measurements. The fitness dataset from Rawlings (1998) contains
measurements of the following seven variables obtained from 31 men:

• Age: Age in years;\\
• Weight: Weight in kg;\\
• Oxygen: Oxygen intake rate, ml per kg body weight per minute;\\
• RunTime: time to run 1.5 miles in minutes ; • RestPulse: heart rate
while resting;\\
• RunPulse: heart rate at end of run;\\
• MaxPulse: maximum heart rate recorded while running;

From the data set fitness.csv a linear model (predicting Oxygen) will be
developed. The bootstrapping function used to provide confidence
intervals came from an original function provided by Donovan (2018) and
improved at a later stage. The current report uses R 3.5.1 software (R
Core Team, 2018).

A linear model was fitted in each analysis and the bootstrap was used to
generate confidence intervals for each of the covariates of interest.
Conclusions to hypothesis tests regarding the significance of the
relationships between the response and the parameter estimates can be
drawn using bootstrap methods. If the confidence interval contains zero,
one fails to reject the null hypothesis, and if it does not contain
zero, one can reject the null hypothesis.

This is an R Markdown document. Markdown is a simple formatting syntax
for authoring HTML, PDF, and MS Word documents. For more details on
using R Markdown see \url{http://rmarkdown.rstudio.com}.

When you click the \textbf{Knit} button a document will be generated
that includes both content as well as the output of any embedded R code
chunks within the document. You can embed an R code chunk like this:

\begin{verbatim}
##      speed           dist       
##  Min.   : 4.0   Min.   :  2.00  
##  1st Qu.:12.0   1st Qu.: 26.00  
##  Median :15.0   Median : 36.00  
##  Mean   :15.4   Mean   : 42.98  
##  3rd Qu.:19.0   3rd Qu.: 56.00  
##  Max.   :25.0   Max.   :120.00
\end{verbatim}

\subsection{Including Plots}\label{including-plots}

You can also embed plots, for example:

\includegraphics{180028841_mt5763_indivudual_report_files/figure-latex/pressure-1.pdf}

\textbar{} Note that the \texttt{echo\ =\ FALSE} parameter was added to
the code chunk\\
to prevent printing of the R code that generated the plot.


\end{document}
